\documentclass[11pt]{article}
\usepackage[letterpaper,margin=1in]{geometry}
\usepackage{amsmath}
\usepackage{amssymb}
\usepackage{amsthm}
%\usepackage{natbib}
%\usepackage{graphicx}
\usepackage{color}
\newcommand{\alert}{\textcolor{red}}
\newcommand{\comm}{\textcolor{blue}}
\usepackage[pdfstartview=FitH,pdfpagelayout=OneColumn]{hyperref}
\usepackage{url}
% Minimal template only needs the above


\begin{document}
		
\section{Abstract}\\
\par The Federal Aviation Administration (FAA) is an agency within the U.S Department of Transportation. Scheduling of the roughly 14,000 air traffic controllers (ATC) that should be assigned to 316 air traffic control facilities, is one of the FAA's roles. In order to schedule, managers are using Operational Planning and Scheduling (OPAS). But, OPAS can only schedule standardly for shifts that are 8-hour/5 days-a-week work, and ATCs have Alternative Work Schedules (AWSs) too. So, the FAA asked a group of students named Senior Design Team (SDT) to create a scheduling methodology for scheduling AWSs, with the objective to minimize overtime hours. However, after formulating the problem, solving the model and running the respective code in Python for 24 hours, the $MIP \  Gap$ was still  15.50\% which wasn't good. In this project, the goal is finding a way to make the problem of ATC Scheduling easier to solve, or being able to find a better solution.
\\
\\
\indent{\textbf{Key words:} }Federal Aviation Administration (FAA), Operational Planning and Scheduling (OPAS), Alternative Work Schedules (AWSs) , air traffic controllers (ATC)
\\
\\

\section{Research proposal}
\\

\indent{\textbf {Background and context}}\\

The Federal Aviation Administration (FAA) is an agency within the U.S Department of Transportation, which was founded in August 1958 as the replacement of the former Civil Aeronautics Administration (CAA). It has the power to construct and operate  airports, manage the air traffic, certificate personnel and aircraft, and  protect the U.S. assets during the launch or re-entry of commercial space vehicles. One of the major role's of the FAA is to develop and operate the system of air traffic control, and navigate civil and military aircraft. In this project, we want to focus specifically on one role that the FAA has, and that role is scheduling of the roughly 14,000 air traffic controllers (ATC) who need to be assigned to 316 air traffic control facilities. \\
\\

\noindent{\textbf {Problem statement}}\\

At the FAA, it is the facility managers' duty to establish the two-week work schedules for the employees. For now, managers are using Operational Planning and Scheduling (OPAS) for scheduling. However, the main problem is that OPAS can only be used for standard schedules which are 8-hour/5 days-a-week work, and the FAA, lets the ATCs have Alternative Work Schedules (AWSs) too. So, OPAS can not be used for creating the AWSs, and managers have to use OPAS for scheduling, and also they have to  create these AWSs manually. These manual adjustments, are affecting the overtime costs. Actually, according to Dr. Nina Barker, Program Manager at the FAA, and Mr. Brandon Kreutel, OPAS team member, in the past three years the overtime cost of ATCs across all facilities  has increased 30\% cumulatively every year. 
 \\

As OPAS couldn't create schedules for AWSs, the FAA asked a group of students named Senior Design Team (SDT) for help, to create a scheduling methodology that can create schedules for AWSs, and the goal was to minimize overtime hours. The Senior Design Team created a methodology for making schedules which minimizes overtime hours scheduled and meets the demand and other scheduling requirements at the same time.  \\

\noindent{\textbf {Importance of the project}}\\

According to the the Senior Design Team reports, files and data that they gathered, with the help of Dr. Balasundaram Baski, they were successful at developing and implementing two mathematical formulations for the shift distribution and ATC scheduling problems. They used the Python programming language for coding, and they solved the problem using $ Gurobi^{®}$ Optimization Software. They also used the Excel to visualize the schedules which were generated by the $Gurobi^{®}$ solver. The “Cowboy” supercomputer at the High-Performance Computing Center (HPCC) at Oklahoma State University was used for solving the ATC Scheduling Model (ATCSM). However, After solving the model and running the code for 24 hours, the MIP Gap was still  15.50\% which wasn't good. In fact, scheduling formulations were too complicated that the SDT was unable to consider all the constraints, too. They mentioned in their report file: "Whether it be through use of a PhD student at a university, a consulting service, or further study within the FAA, the SDT believes that exploring adjustments to the formulation would be beneficial in the future."  
 \\
 \\
 \noindent{\textbf {Objectives and Methods}}\\
\\
 In this project, the goal is searching for some ways so the problem of ATC Scheduling could be solved faster, and that a better solution could be found. In order to that, through this project, several alternatives will be considered: \\
 
\indent  $\bullet$ Considering some adjustments in the mathematical formulation of the problem to make it easier to solve\\

 \indent $\bullet$ Searching for a heuristic so that the existing  formulation for the problem can be solved faster and get better results\\
 
 \indent $\bullet$ Trying to reformulate the problem in another way that can be solved easier.(e.g. using the binary variable of the min up\down type.) \\
 \\
 \noindent{\textbf {Data source}}\\

 For this project, we have 5 large-scale data sets that Dr. Balasundaram Baski has received as part of the project of SDT with the FAA; and those data sets are from "Jacksonville Air Route Traffic Control Center". \\

\noindent{\textbf {Formulations of the problem}}\\

\\
In the following, in the third section, problem parameters and notations used are listed.Also, we will see major modeling assumptions in the fourth section, and the MIP formulation of the problem in the last two sections. All these sections has developed by Dr. Balasundaram Baski.


\newpage

\section{Problem parameters and notations used}
\begin{itemize}
\item The unit of time measurement is a ``slot'' that is 15 minutes long; the biweekly pay-period is the planning horizon, denoted $1,\ldots,T$ corresponding to $T = 4\times 24 \times 14 = 1344$ slots.

\item The \#ATCs needed (staffing requirement) in each slot $k$ is given by $r_k$, where $k = 1,\ldots,T$.

\item The \#types of shifts allowed, where a shift type is based on the duration of the shift, is denoted by $L$; in the present situation, $L=3$, as 8-hr, 9-hr, and 10-hr shifts are the only allowable types. 

\item The duration of a shift of type $\ell$ is denoted by $d_\ell$, for each $\ell =1,\ldots,L$, such that $d_1 < d_2 < \cdots < d_L$; in the present situation, $d_1 = 8 \times 4 = 32, d_2 = 9 \times 4 = 36, d_3 = 10 \times 4 = 40.$

\item The minimum duration of a shift is $d_{\min}$ and maximum duration is $d_{\max}$. Typically, $d_{\min} = d_1$ and $d_{\max} = d_L$.

\item A schedule of type $\ell$ is made up of $n_\ell$ shifts of type $\ell$,  for each $\ell = 1,\ldots,L$; i.e. a regular schedule consists of 10 shifts, each 8-hrs long, so $n_1 = 10$; an alternate schedule consists of 9 shifts, each 9-hrs long,\footnote{Review the simplifying assumption made regarding these alternate work schedules.} so $n_2 = 9$; a compressed schedule consists of 8 shifts, each 10-hrs long, so $n_3 = 8$. 

\item The \#schedules of each type $\ell$ is \underline{assumed to be given} by the user, denoted as $m_\ell$ for  $\ell =1,\ldots,L$; e.g. the facility manager must specify that 30 regular 8-hr schedules ($m_1 = 30$), 20 alternate 9-hr schedules ($m_2 = 20$), and 15 compressed 10-hr schedules ($m_3 = 15$) need to be constructed.

\item The minimum resting delay between two consecutive shifts in any schedule is assumed to be a given constant $\Delta$. 

\end{itemize}

\section{Major modeling assumptions} 
\begin{itemize}
\item Breaks are scheduled by the facility manager during implementation and are not planned by the model. 

\item In reality, the minimum delay $\Delta$  depends on whether the majority of the shift is daytime, evening, or night; we make a simplifying assumption removing this dependence and assuming it to be a constant number (9 hours is used as a default value in our experiments).  

\item All schedules created must begin and end inside the planning horizon \underline{excluding} the minimum resting delay $\Delta$ after the last shift, i.e. no schedule or overtime shift can span across two different pay-periods. Since the minimum resting delay after the last shift cannot be enforced inside the planning horizon, this must be accounted for when schedules are assigned to employees. 

\item A shift is not allowed to start during certain  times of the day. For instance, 12am--5am could be such a prohibited time-window. However, in the current version of the formulation, this requirement is not enforced. 

\item A shift can overrun, but the amount of time by which the shift duration exceeds its nominal duration is considered overtime. A shift of type $\ell$ is nominally of duration $d_\ell$, and no shift, including one that overruns, can exceed the maximum duration of a shift, which is $d_{\max}$.

\item An ATC not regularly scheduled could be called in to do an overtime shift; such a shift must have a minimum duration of $d_{\min}$ and a maximum duration of $d_{\max}$. 

\item An alternate schedule, in reality consists of 8 shifts, each 9-hrs long and one shift that is 8-hrs long; we make a simplifying assumption that it consists of 9 shifts, each 9-hrs long; so the facility manager must decide during implementation, which one of these 9-hr shifts can be truncated to 8-hrs without affecting demand coverage, or may have to treat it as a shift overrun, i.e. overtime. 

\item The number of schedules of each type sought is also assumed to be an input; the facility manager could base this on historical data, experience, and/or preferences expressed by the ATCs. 

\end{itemize}

\section{Shift distribution model}
The following formulation can be used to determine the number of shifts of each type  that are needed to cover the staffing requirements in a pay-period. The model is designed to minimize the total duration of all the shifts used.

\paragraph{Decision variables.} Let $x_i^\ell$ denote the number of shifts of type $\ell$ that must start in time-slot $i$, for each $\ell = 1,\ldots,L$, and $i = 1,\ldots, T-d_\ell+1$.

\begin{align}
\label{eq.SDobj}\textbf{(SDO)}\quad\min \sum_{\ell =1}^L\sum_{i=1}^{T-d_\ell+1}d_\ell x_i^\ell & \\
\label{eq.SDcon} \text{subject to.}  \quad  \sum_{\ell=1}^L\sum_{i=\max\{1,k-d_\ell+1\}}^{\min\{k,T-d_\ell+1\}}x_i^\ell &\ge r_k \quad \forall k =1,\ldots,T\\
x_i^\ell & \ge 0 \text{ and integer } \forall \ell = 1,\ldots,L;  i = 1,\ldots, T-d_\ell+1
\end{align}

The objective~(\ref{eq.SDobj}) is the total number of slots across all shifts that need to be worked to cover the staffing requirements in the given planning horizon. Constraints~(\ref{eq.SDcon}) state that the number of shifts of each type that cover any given time-slot must be large enough to meet the staffing requirement in that slot. Since no schedules are constructed by this model, that step has to be manually completed by the facility manager. 

\section{ATC scheduling model}

\paragraph{Decision variables.} 
\begin{itemize}
\item Let $s_{ij}^\ell$ and $t_{ij}^\ell$ denote the start time and completion time, respectively, of the $i$-th shift in the $j$-th schedule of type $\ell$, where $\ell = 1,\ldots, L; i = 1,\ldots,n_\ell; j=1,\ldots,m_\ell$. 
\item Let $z_{ijk}^\ell$ be a binary  variable that indicates if the $i$-th shift in the $j$-th schedule of type $\ell$ covers time-slot $k$, where $\ell = 1,\ldots, L; i = 1,\ldots,n_\ell; j=1,\ldots,m_\ell; k=1\ldots,T$.
\item Let $o_{uv}$ denote the number of ATCs working an overtime shift from time-slot $u$ to time-slot $v$ (both inclusive), where $u= 1,\ldots, T-d_{\max}+1; v =  u+d_{\min}-1, \ldots, u+d_{\max}-1$.
\end{itemize}
%\sum_{u=\max\{1,k-d_L+1\}}^k\sum_{v=k}^{k+d_L-1}
\begin{align}
\label{eq.ATCSobj}\textbf{(ATCSO)}\quad\min \sum_{\ell=1}^L\sum_{i=1}^{n_\ell}\sum_{j=1}^{m_\ell}(t_{ij}^\ell-s_{ij}^\ell-d_\ell) &+ \sum_{u=1}^{T-d_L+1}\sum_{v=u+d_1-1}^{u+d_L-1}(v-u+1)\times o_{uv}\\
\label{eq.ATCSnextshiftcon}\text{subject to.} \quad s_{i+1,j}^\ell  \ge t_{ij}^\ell + \Delta &\quad \forall \ell = 1,\ldots,L; i=1,\ldots,n_\ell-1;\\
\nonumber &\quad\  j=1,\ldots, m_\ell\\
\label{eq.ATCSminshiftlength}t_{ij}^\ell - s_{ij}^\ell \ge d_\ell   &\quad \forall \ell = 1,\ldots,L; i=1,\ldots,n_\ell; \\
\nonumber &\quad\ j=1,\ldots, m_\ell\\
\label{eq.ATCSmaxshiftlength}t_{ij}^\ell - s_{ij}^\ell \le d_{\max}   &\quad \forall \ell = 1,\ldots,L; i=1,\ldots,n_\ell; \\
\nonumber & \quad\ j=1,\ldots, m_\ell\\
\label{eq.ATCSdemandcon} \sum_{\ell=1}^L\sum_{i=1}^{n_\ell}\sum_{j=1}^{m_\ell}z_{ijk}^\ell + \sum_{\substack{u=1,\ldots, T-d_{\max}+1\\ v =  u+d_{\min}-1, \ldots, u+d_{\max}-1\\ u \le k \le v}}o_{uv}\ge r_k  &\quad \forall k=1,\ldots,T\\
\label{eq.ATCSstartb4k}s_{ij}^\ell \le k\times z_{ijk}^\ell + [T-(n_\ell-i+1)(d_\ell + \Delta)]\times(1-z_{ijk}^\ell) &\quad \forall \ell = 1,\ldots,L; i=1,\ldots,n_\ell; \\
\nonumber &\quad \ j=1,\ldots, m_\ell; k=1,\ldots,T\\
\label{eq.ATCSendafterk}t_{ij}^\ell \ge k\times z_{ijk}^\ell + i\times(d_\ell+\Delta)\times(1-z_{ijk}^\ell) &\quad \forall \ell = 1,\ldots,L; i=1,\ldots,n_\ell; \\
\nonumber &\quad \ j=1,\ldots, m_\ell; k=1,\ldots,T\\
%\label{eq.ATCSstartbounds}s_{ij}^\ell \le T - (n_\ell-i + 1)\times(d_\ell + \Delta)    &\quad \forall \ell = 1,\ldots,L; i=1,\ldots,n_\ell; \\
%\nonumber &\quad\ j=1,\ldots, m_\ell\\
%\label{eq.ATCSfinishbounds}t_{ij}^\ell  \le T - (n_\ell-i)\times(d_\ell + \Delta)  &\quad \forall \ell = 1,\ldots,L; i=1,\ldots,n_\ell; \\
%\nonumber & \quad\ j=1,\ldots, m_\ell\\
\label{eq.ATCSdisaggZ}\sum_{i=1}^{n_\ell}z_{ijk}^\ell \le 1 &\quad  \forall \ell = 1,\ldots,L; j=1,\ldots, m_\ell; \\[-3mm]
\nonumber &\quad\ k=1,\ldots,T\\
\label{eq.lastshiftbound} t_{n_\ell,j}^\ell \le T &\quad \forall \ell = 1,\ldots,L; j=1,\ldots, m_\ell\\
\label{eq.ATCSNN} s_{ij}^\ell, t_{ij}^\ell  \ge 0 \text{ and integer}  &\quad \forall \ell = 1,\ldots,L; i=1,\ldots,n_\ell; \\
\nonumber & \quad\ j=1,\ldots, m_\ell
\end{align}

The ATCSO model aims to build $m_\ell$ schedules of type $\ell = 1,\ldots,L$ that minimizes the total shift overruns and overtime shifts while simultaneously meeting the staffing requirements over the planning horizon. The first term in objective~(\ref{eq.ATCSobj}) measures total shift overruns and the second term measures total duration of overtime shifts created to cover staffing requirements in the planning horizon. 

Constraints~(\ref{eq.ATCSnextshiftcon}) require that the $(i+1)$-th shift in a schedule starts after the $i$-th shift finishes and the minimum required resting delay has elapsed. The quantity $t_{ij}^\ell - s_{ij}^\ell$ is the duration of the $i$-th shift in the $j$-th schedule of type $\ell$. Constraints~(\ref{eq.ATCSminshiftlength}) require this to be at least $d_\ell$ since this is a shift of type $\ell$. Since overruns are allowed, the shift duration could exceed $d_\ell$, the excess amount is minimized in the first term in objective~(\ref{eq.ATCSobj}). However, it cannot exceed the maximum possible length of any shift, which is $d_{\max}$; this is enforced by constraints~(\ref{eq.ATCSmaxshiftlength}). Note that we require the shift types to be labeled in the increasing order of the shift duration. 

Constraints~(\ref{eq.ATCSdemandcon}) enforce the staffing requirements driving this problem. 
 The first term captures the following: the binary variable $z_{ijk}^\ell$ is used to indicate that the $i$-th shift in the $j$-th schedule of type $\ell$ covers the time-slot $k$. If so, it counts the one ATC working that shift. Subsequent constraints~~(\ref{eq.ATCSstartb4k}--\ref{eq.ATCSendafterk}) enforce this behavior on the $z$-variables. The second term   counts all the possible overtime shifts that cover time-slot $k$. Recall that $o_{uv}$ is the number of ATCs working overtime shift from time-slot $u$ to time-slot $v$. For an overtime shift starting in time-slot $u$ and ending in time-slot $v$ to cover $k$, we require $u \le k \le v$. The duration of this overtime shift is $v-u+1$ time slots, which must be at least $d_{\min}$ and at most $d_{\max}$. This overtime shift contributes $o_{uv}$ ATCs towards the staffing requirement. 

The role we seek the binary variable $z_{ijk}^\ell$ to play in the model is enforced using constraints~(\ref{eq.ATCSstartb4k}--\ref{eq.ATCSendafterk}). Suppose $z_{ijk}^\ell = 1$ in some feasible solution to the model; these constraints then reduce to stating that $s_{ij}^\ell \le k \le t_{ij}^\ell$, i.e., the $i$-th shift in the $j$-th schedule of type $\ell$ covers time-slot $k$, as required. If $z_{ijk}^\ell = 0$, these constraints become redundant, as explained next. In this case constraint~(\ref{eq.ATCSstartb4k}) reduces to stating $s_{ij}^\ell \le  [T-(n_\ell-i+1)(d_\ell + \Delta)]$. The quantity on the right measures the minimum amount of time that must elapse after the $i$-th shift starts; $n_\ell-i+1$ shifts followed by the resting delay $\Delta$ must follow the start of the $i$-th shift. Hence, the quantity on the right marks the latest point in time the $i$-th shift in the schedule can start. Similarly, constraint~(\ref{eq.ATCSendafterk}) reduces to $t_{ij}^\ell \ge i\times(d_\ell+\Delta)$ if $z_{ijk}^\ell = 0$. Again, at least $ i\times(d_\ell+\Delta)$ must have elapsed before the $i$-th shift completes, and therefore this constraint is also redundant if  $z_{ijk}^\ell = 0$. Finally, we note that if  $z_{ijk}^\ell = 0$, we do not force the condition that $k$ must be outside the interval $[s_{ij}^\ell,t_{ij}^\ell]$. This is unnecessary as it does not affect the correctness of feasible solutions to this model. Suppose $z_{ijk}^\ell = 0$ and $s_{ij}^\ell \le k \le t_{ij}^\ell$ in a feasible solution, we  underestimate the number of ATCs actually present during time-slot $k$ in constraint~(\ref{eq.ATCSdemandcon}). But the staffing requirement is indeed met, since the solution is feasible to the model. Hence, the schedules built by solving the model that are reported to the user must be used to count the actual coverage in each time-slot instead of using the $z$-variables for this purpose. 

Constraints~(\ref{eq.ATCSdisaggZ}) are added to strengthen the formulation; they say that in any given schedule, at most one of the shifts could cover any given time-slot as the shifts are disjoint intervals in time. 
Constraint~(\ref{eq.lastshiftbound}) places an upper-bound on the completion time of the last shift in every schedule. This in turn imposes finite bounds on all start time and completion time variables of the previous shifts through constraints~(\ref{eq.ATCSnextshiftcon}--\ref{eq.ATCSmaxshiftlength}).

%    \bibliography{refs}
%    \bibliographystyle{plainnat}
%\bibliographystyle{plain}
\end{document}
